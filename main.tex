\documentclass[a4paper, article, oneside, UKenglish]{report}


%% Title page
\usepackage[UKenglish]{uiomasterfp}


%% Encoding
\usepackage[utf8]{inputenx} % Source code
\usepackage[T1]{fontenc}    % PDF


%% Fonts and typography
\usepackage{lmodern}           % Latin Modern Roman
\usepackage[scaled]{beramono}  % Bera Mono (Bitstream Vera Sans Mono)
\renewcommand{\sfdefault}{phv} % Helvetica
\usepackage[final]{microtype}  % Improved typography
\renewcommand{\abstractnamefont}{\sffamily\bfseries}                 % Abstract
\renewcommand*{\chaptitlefont}{\Large\bfseries\sffamily\raggedright} % Chapter
\setsecheadstyle{\large\bfseries\sffamily\raggedright}               % Section
\setsubsecheadstyle{\large\bfseries\sffamily\raggedright}            % Subsection
\setsubsubsecheadstyle{\normalsize\bfseries\sffamily\raggedright}    % Subsubsection
\setparaheadstyle{\normalsize\bfseries\sffamily\raggedright}         % Paragraph
\setsubparaheadstyle{\normalsize\bfseries\sffamily\raggedright}      % Subparagraph

%% Mathematics
\usepackage{amssymb}   % Extra symbols
\usepackage{amsthm}    % Theorem-like environments
\usepackage{thmtools}  % Theorem-like environments
\usepackage{mathtools} % Fonts and environments for mathematical formuale
\usepackage{mathrsfs}  % Script font with \mathscr{}


%% Miscellaneous
\usepackage{graphicx}  % Tool for images
\graphicspath{{figures/}}
\usepackage{babel}     % Automatic translations
\usepackage{csquotes}  % Quotes
\usepackage{textcomp}  % Extra symbols
\usepackage{listings}  % Typesetting code
\lstset{basicstyle = \ttfamily, frame = tb}
\usepackage{enumitem}
\usepackage{tikz-cd}
\usepackage{spectralsequences}
\usepackage{titling}


%% Bibliography
\usepackage{mathscinet}
\usepackage[backend    = biber,
            sortcites  = true,
            giveninits = true,
            doi        = false,
            isbn       = false,
            url        = false,
            sortlocale = nb_NO,
            style      = alphabetic]{biblatex}
\DeclareNameAlias{sortname}{family-given}
\DeclareNameAlias{default}{family-given}
\DeclareFieldFormat[article]{volume}{\bibstring{jourvol}\addnbspace#1}
\DeclareFieldFormat[article]{number}{\bibstring{number}\addnbspace#1}
\renewbibmacro*{volume+number+eid}
{
    \printfield{volume}
    \setunit{\addcomma\space}
    \printfield{number}
    \setunit{\addcomma\space}
    \printfield{eid}
}
\addbibresource{bibliography.bib}


%% Cross references
\usepackage{varioref}
\usepackage[pdfusetitle]{hyperref}
\urlstyle{sf}
\usepackage[nameinlink, capitalize, noabbrev]{cleveref}
\crefname{chapter}{Section}{Sections}


%% Theorem-like environments
\declaretheorem[style = plain, numberwithin = chapter]{theorem}
\declaretheorem[style = plain,      sibling = theorem]{corollary}
\declaretheorem[style = plain,      sibling = theorem]{lemma}
\declaretheorem[style = plain,      sibling = theorem]{proposition}
\declaretheorem[style = definition, sibling = theorem]{definition}
\declaretheorem[style = definition, sibling = theorem]{example}
\declaretheorem[style = remark,    numbered = no]{remark}


%% Delimiters
\DeclarePairedDelimiter{\p}{\lparen}{\rparen}   % Parenthesis
\DeclarePairedDelimiter{\set}{\lbrace}{\rbrace} % Set
\DeclarePairedDelimiter{\abs}{\lvert}{\rvert}   % Absolute value
\DeclarePairedDelimiter{\norm}{\lVert}{\rVert}  % Norm


%% Operators
\newcommand{\diff}{\mathop{}\!\mathrm{d}}
\DeclareMathOperator{\im}{im}
\DeclareMathOperator{\rank}{rank}
\DeclareMathOperator{\E}{E}
\DeclareMathOperator{\Var}{Var}
\DeclareMathOperator{\Cov}{Cov}


%% New commands for sets
\newcommand{\N}{\mathbb{N}}   % Natural numbers
\newcommand{\Z}{\mathbb{Z}}   % Integers
\newcommand{\Q}{\mathbb{Q}}   % Rational numbers
\newcommand{\R}{\mathbb{R}}   % Real numbers
\newcommand{\C}{\mathbb{C}}   % Complex numbers
\newcommand{\A}{\mathbb{A}}   % Affine space
\renewcommand{\P}{\mathbb{P}} % Projective space


%% New commands for vectors
\renewcommand{\a}{\mathbf{a}}
\renewcommand{\b}{\mathbf{b}}
\renewcommand{\c}{\mathbf{c}}
\renewcommand{\v}{\mathbf{v}}
\newcommand{\w}{\mathbf{w}}
\newcommand{\x}{\mathbf{x}}
\newcommand{\y}{\mathbf{y}}
\newcommand{\z}{\mathbf{z}}
\newcommand{\0}{\mathbf{0}}
\newcommand{\1}{\mathbf{1}}


%% Miscellaneous
\renewcommand{\qedsymbol}{\(\blacksquare\)}

\usepackage{subfiles}

\title{Rational Homotopy Groups of Spheres}
\author{Mika Bohinen}
% Multiple supervisors: \supervisor{Supervisor 1}{Supervisor 2}...{Supervisor n}
% Skip supervisor for MAT2500

\begin{document}

\uiomasterfp[long=10,
             dept={Department of Mathematics},
             fac={},
             program={MAT2000},
             kind={Project work},
             color=blue,
             supervisor={John Christian Ottem}
]


\begin{abstract}
    \noindent
    Homotopy groups of spheres are generally quite difficult to compute,
    and they can behave quite erratic. We only focus on the non-torsion
    part of these groups and aim to show that for \( n \) even, the
    homotopy groups of \( S^n \) are torsion except for the \( n \)th and
    \( 2n-1 \)th group, while for \( n \) odd, the homotopy groups of
    \( S^n \) are torsion except for the \( n \)th group. The proof
    relies on finding a rational homotopy equivalence between \( S^n \) 
    and \( K(\Z, n) \) in the odd case and between \( S^n \) and
    a certain homotopy fibre \( F \) in the even case. We establish
    this link by using the Leray-Serre spectral sequence to 
    calculate the rational cohomology algebra of \( K(\Z, n) \) in
    the odd case and \( F \) in the even case.
    

\end{abstract}
\section*{\centering{Acknowledgments}}
I am incredibly grateful to my advisor,
John Christian Ottem, who
guided me through this project and suggested such an exciting
topic to write about. Thanks should also go to the organizers of the
course MAT2000 at UiO for providing this excellent opportunity and making it such
a memorable experience. Lastly, I would be remiss in not mentioning the
support from friends and family, especially my dad, who has always
encouraged my endeavours in mathematics.
\pagenumbering{roman}

\tableofcontents
\pagenumbering{arabic}

\chapter{Introduction}
\subfile{chapters/chapter1}

\chapter{Basic Constructions}
\subfile{chapters/chapter2}

\chapter{Spectral sequences and the rational cohomology algebra of Eilenberg-MacLane spaces}
\subfile{chapters/chapter4}

\chapter{Proof of rational homotopy groups of spheres}
\subfile{chapters/chapter5}
\printbibliography


\end{document}
