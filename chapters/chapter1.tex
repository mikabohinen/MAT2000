\documentclass[../main.tex]{subfiles}
\graphicspath{{\subfix{./figures/}}}
\begin{document}
Homotopy groups of spheres are notoriously difficult to compute. This is in
stark contrast to the homology groups of spheres that are trivial except
for the zeroth dimension and the dimension of the sphere itself.
Heinz Hopf provided in 
\cite[]{Hop31} the first example of this difference via the construction
of a non-trivial mapping \( p:S^3\rightarrow S^2 \).
Indeed, this mapping is a generator of the infinite cyclic group
\( \pi_3(S^2) \), as shown in \cref{ex:hopf}. Considering this, one would very much like to be able
to compute \( \pi_{n+k}(S^n) \) for arbitrary \( k,n>0 \). This is
no easy task and has fueled the development of powerful 
tools in algebraic topology.

However, there is a much more definitive answer if one ignores torsion and
takes the tensor product with \( \Q \). The goal of this paper is to
show that for \( n \) even, we have
\begin{equation}
    \label{eq:even-spheres}
    \pi_k(S^n)\otimes \Q \cong
    \begin{cases}
        \Q, &\text{ if }k=n \text{ or } k=2n-1\\
        0 & \text{ otherwise}
    \end{cases}
\end{equation}
while for \( n \) odd, we have
\begin{equation}
    \label{eq:odd-spheres}
    \pi_k(S^n)\otimes \Q \cong
    \begin{cases}
        \Q, &\text{ if }k=n\\
        0, &\text{ otherwise.}
    \end{cases}
\end{equation}
The main ideas of the proof are taken from \cite[]{Ber12}. We
will use the rational cohomology algebra of the \emph{Eilenber-Maclane
spaces}
 \( K(\Z, n) \) to find \emph{rational homotopy equivalences} between
 the spheres and some spaces we already know the rational homotopy
 groups of.  

Chapter 2 introduces some of the main concepts, including
\emph{Serre fibration} and the aforementioned Eilenberg-MacLane
spaces. We also introduce \emph{homotopy fibres} and compute the
homotopy groups of some homotopy fibres of Eilenberg-MacLane spaces which
will be necessary for the proof when \( n \) is even. At the end of
Chapter 2, we state equivalent conditions for a rational homotopy equivalence
which shall allow us to exploit the algebra structure of the rational
cohomology of \( K(\Z, n) \).
Now, to compute \( H^*(K(\Z, n);\Q) \), we introduce, in Chapter 3,
\emph{spectral sequences} and state the Leray-Serre spectral sequence,
which allows us to perform the computations. Then, using some
of the properties of \( K(\Z, n) \) introduced in Chapter 2, we find,
in Chapter 4, the rational homotopy equivalences, which allow
us to show the results in \cref{eq:even-spheres} and \cref{eq:odd-spheres}.

\end{document}
