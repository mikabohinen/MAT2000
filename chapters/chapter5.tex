\documentclass[../main.tex]{subfiles}
\graphicspath{{\subfix{./figures/}}}
\begin{document}
Having computed the rational cohomology algebra of \( K(\Z, n) \), we are
now in a position to compute the rational homotopy groups of spheres.
We shall concentrate on the even and odd-dimensional cases separately,
and the main ideas of the proof are taken from \cite[]{Ber12}. The
objective in the odd case is to show that there is a rational homotopy
equivalence between \( K(\Z, n) \) and \( S^n \), and
a rational homotopy equivalence between a certain homotopy fibre
\( F \) and \( S^n \) in the even case.

\section{Odd-dimensional spheres}
\begin{theorem}[Rational homotopy groups of odd spheres]
    Let \( n\in \N \) be odd, then
    \begin{equation}
        \label{eq:odd}
        \pi_k(S^n)\otimes \Q =
        \begin{cases}
            \Q, &\text{ if } k = n\\
            0, &\text{ otherwise.}
        \end{cases}
    \end{equation}
\end{theorem}
\begin{proof}
If we can find an isomorphism \( f^*:H^*(K(\Z, n);\Q) \rightarrow 
H^*(S^n; \Q)\) then by \cref{prop:rat-cohomology} we have that
\( f_*:\pi_*(S^n)\otimes \Q \rightarrow \pi_*(K(\Z, n))\otimes\Q \) is
an isomorphism and we will be done. We therefore choose \( f:S^n
\rightarrow K(\Z, n) \) to represent the
generator of \( \pi_n(K(\Z, n))\). The induced map \( f_*:\pi_n(S^n)
\rightarrow \pi_n(K(\Z, n))\) must then be an isomorphism. From the
naturality of the Hurewicz homomorphism and Corollary 1.2.1. in \cite[]{Moe15} 
we have a commutative diagram:
\begin{equation}
    \label{eq:big-commutative}
    \begin{tikzcd}
        {\pi_n(S^n)\otimes \Q} & {} & {\pi_n(K(\Z,n))\otimes \Q} \\
        {H_n(S^n,\Q)} && {H_n(K(\Z,n),\Q)} \\
        {\text{Hom}_\Q(H_n(S^n,\Q),\Q)} && {\text{Hom}_\Q(H_n(K(\Z,n),\Q),\Q)} && {} \\
        {H^n(S^n;\Q)} && {H^n(K(\Z,n);\Q)}
        \arrow["{f_*\otimes\text{id}}", from=1-1, to=1-3]
        \arrow["{f_*}", from=2-1, to=2-3]
        \arrow["\cong"', from=1-1, to=2-1]
        \arrow["\cong", from=1-3, to=2-3]
        \arrow["\cong"', from=2-1, to=3-1]
        \arrow["\cong", from=2-3, to=3-3]
        \arrow["{f^*}", from=3-3, to=3-1]
        \arrow["{f^*}", from=4-3, to=4-1]
        \arrow["\cong", from=3-3, to=4-3]
        \arrow["\cong"', from=3-1, to=4-1]
    \end{tikzcd}
\end{equation}
Since all the vertical maps and the top map are isomorphisms, it follows
that the horizontal maps are isomorphisms.
Hence, it follows that \( f^*:H^n(K(\Z, n);\Q)\rightarrow H^n(S^n; \Q ) \) 
is an isomorphism.
Then, since both
\( H^n(K(\Z, n);\Q) \) and \( H^n(S^n, \Q) \) are exterior algebras on
a single generator in degree \( n \), we must have that \( f^* \)
extends to an isomorphism in all degrees so that \( f^*:H^*(K(\Z, n);
\Q)\rightarrow H^*(S^n;\Q)\) is an isomorphism. Then, as remarked earlier,
\cref{prop:rat-cohomology} implies that \( f_*:\pi_*(S^n)\otimes \Q
\rightarrow \pi_*(K(\Z, n))\otimes\Q\) is an isomorphism and hence
we are done.
\end{proof}

\section{Even-dimensional spheres}
The rational cohomology algebra of \( K(\Z, n) \) for even \( n \) is,
by \cref{thm:eil-mac}, isomorphic to \( \Q[x] \). There is therefore
no map \( f:S^n \rightarrow K(\Z, n) \) which induces an isomorphism
in cohomology and we cannot use the same strategy as in the odd
dimensional case and we must take another approach.


Let \( g:K(\Z, n)\rightarrow K(\Z, 2n) \) represents the class
\( x^2\in H^{2n}(K(\Z, n))\cong [K(\Z, n), K(\Z, 2n)] \) and
\( F \) the homotopy fibre of \( g \).
\begin{lemma}
    The rational cohomology algebra of \( F \) is isomorphic to 
    \( \Lambda_\Q(x) \) with \( \abs{x}=n \).
\end{lemma}
\begin{proof}
   From \cref{prop:fibres} we know that the homotopy fibre of
   \( F \) with its projection map to \( K(\Z, n) \) is homotopically
   equivalent to \( K(\Z, 2n-1)\simeq \Omega K(\Z,2n) \). If we
   therefore also consider the path space fibration of \( K(\Z, 2n) \) 
   we have the following diagram:
   \begin{equation}
       \begin{tikzcd}
            {\Omega K(\Z, 2n)} & {PK(\Z, 2n)} & {K(\Z, 2n)} \\
            {\Omega K(\Z, 2n)} & F & {K(\Z, n)}
            \arrow[from=2-1, to=2-2]
            \arrow[from=2-2, to=2-3]
            \arrow[from=1-2, to=1-3]
            \arrow[from=1-1, to=1-2]
            \arrow["{\text{id}}", tail reversed, from=2-1, to=1-1]
            \arrow[from=2-2, to=1-2]
            \arrow["g"', from=2-3, to=1-3]
        \end{tikzcd}
   \end{equation}
    where the right square commutes and the left square commutes up to
    homotopy. Letting \( A \) denote the spectral sequence induced
    by the fibration in the top row and \( B \) the spectral sequence
    in the bottom row, we also have an induced map \( \alpha:A\rightarrow
    B\). From \cref{thm:sss} we know that
    \begin{equation}
        A^{p,q}_2=H^p(K(\Z, 2n-1);\Q)\otimes H^q(K(\Z, 2n);\Q)\Longrightarrow H^{p+q}(PK(\Z, 2n);\Q)=0.
    \end{equation}
    and so, for degree reasons, only \( A^{0, 0}_2=A_{2n}^{0, 0} \) survives on to the \( A_{2n+1
    } \)--page. This means that all the differentials \[ d_{2n}^{2k, 2n-1}
    :A_{2n}^{2k, 2n-1}\rightarrow A_{2n}^{2k+2n, 0}\]
    for \( k\in\N \) are isomorphisms. Of particular interest will be
    the differential \[ d_{2n}^{0, 2n-1}:A_{2n}^{0, 2n-1}\rightarrow
    A_{2n}^{2n, 0}\] since, by the naturality
    of the Serre spectral sequence, and the fact that \( A_2=A_{2n} \),
    \( B_2=B_{2n}\), 
    we have a commutative diagram
    \begin{equation}
        \begin{tikzcd}
            {B_{2n}^{0, 2n-1}} && {B^{2n,0}_{2n}} \\
            \\
            {A^{0, 2n-1}_{2n}} && {A^{2n,0}_{2n}}
            \arrow["{d_{2n}^{0, 2n-1}}", from=1-1, to=1-3]
            \arrow["{d_{2n}^{0, 2n-1}}", from=3-1, to=3-3]
            \arrow["{\text{id}}", from=3-1, to=1-1]
            \arrow["{g^*}"', from=3-3, to=1-3]
        \end{tikzcd}
    \end{equation}
    Now, the bottom map is an isomorphism. So if we can show that
    both of the vertical maps are isomorphisms then the top map must
    also be an isomorphism by the commutativity of the diagram. Since
    \( \text{id} \) is obviously an isomorphism, we need only show that
    \( g^* \) is an isomorphism. Remember
    that the map \( g^*:H^*(K(\Z, 2n))\rightarrow H^*(K(\Z, n)) \) is induced
    from the class \[ x^2\in B^{2n, 0}_{2n}\cong H^{2n}(K(\Z,n))\cong[K(\Z, n), K(\Z, 2n)] \] and
    so if \[ y\in A^{2n, 0}_{2n}\cong H^{2n}(K(\Z, 2n))\cong [K(\Z, 2n),
    K(\Z, 2n)] \] is the class of the identity map \( \iota_{2n}:K(\Z, 2n)\rightarrow K(\Z, 2n) \), then
    \begin{equation}
        g^*(y)=[\iota_{2n}\circ g] = [g]=x^2.
    \end{equation}
    Hence the generator of \( A^{2n, 0}_{2n} \) is taken to the generator
    of \( B^{2n, 0}_{2n} \), so the rightmost map must be an
    isomorphism. 
    Thus, since both vertical maps are isomorphisms as well
    as the bottom map, we must have the top map is also an
    isomorphism. Then, by the graded differential structure of
    \( B_{2n} \) we have that all maps \[ d^{2k, 2n-1}_{2n}:B_{2n}^{2k,2n-1}\rightarrow B_{2n}^{2k+2n, 0} \] are
    isomorphisms. Thus the only terms that survive to the
    \( B_\infty \)--page are \( B_{2n}^{0, 0} \) 
    and \( B_{2n}^{n, 0} \), showing that
    \( B_\infty\cong H^*(F;\Q) \) is an exterior algebra on a single
    generator in degree \( n \) which was what we wanted to show.
\end{proof}

\begin{theorem}[Rational homotopy groups of even spheres]
    Let \( n\in \N \) be even, then
    \begin{equation}
        \pi_k(S^n)\otimes \Q =
        \begin{cases}
            \Q, &\text{ if }k=n, 2n-1\\
            0, &\text{ otherwise.}
        \end{cases}
    \end{equation}
\end{theorem}
\begin{proof}
Let \( f:S^n\rightarrow K(\Z, n) \) denote the same generator as before
and \( g:K(\Z, n)\rightarrow K(\Z, 2n) \) the class \( x^2\in H^{2n}(K(\Z
, n)) \) with \( F \) its homotopy fibre. Since \( \pi_n(K(\Z, 2n))=0 \) 
we have that \( f \) lifts to \( h:S^n\rightarrow F \) making the
following diagram commute:
\begin{equation}
    \begin{tikzcd}
        F & {K(\Z, n)} & {K(\Z, 2n)} \\
        & {S^n}
        \arrow["{g\circ f\simeq *}"', from=2-2, to=1-3]
        \arrow["f", from=2-2, to=1-2]
        \arrow["g", from=1-2, to=1-3]
        \arrow["p", from=1-1, to=1-2]
        \arrow["h", dashed, from=2-2, to=1-1]
    \end{tikzcd}
\end{equation}
We know that \( H^*(F;\Q)\cong \Lambda_\Q(x) \) with \( \abs{x}=n \) 
and so if we can show that \[ h^*:H^*(F;\Q)\rightarrow H^*(S^n; \Q ) \] is an
isomorphisms then we will be done after applying \cref{prop:fibres}
together with \cref{prop:rat-cohomology}. Since \( f \) represents
a generator we have that \[ f_*:\pi_n(S^n)\rightarrow \pi_n(K(\Z, n)) \]
must be an isomorphism. Then, since \( \pi_n(F)\cong\Z \), we have that
\[ h_*:\pi_n(S^n)\rightarrow \pi_n(F) \] is also an isomorphism. By
replacing \( K(\Z, n) \) with \( F \) in (\ref{eq:big-commutative}) we get 
that \( h^*:H^n(F;\Q)\rightarrow
H^n(S^n;\Q)\) is an isomorphism. This means that the generator of
\( H^*(F;\Q) \) is taken to the generator of \( H^*(S^n;\Q) \) which
means \[ h^*:H^n(F;\Q)\rightarrow H^n(S^n;\Q) \] extends to an
isomorphism \( h^*:H^*(F;\Q)\rightarrow
H^*(S^n;\Q)\) and we have found our desired rational
homotopy equivalence concluding the proof.

\end{proof}

\begin{table}[]
    \centering
    \resizebox{\textwidth}{!}{%
    \begin{tabular}{r|c c c c c c c c c c c c c c c c }
       & \( \pi_1 \) & \( \pi_2 \) & \( \pi_3 \) & \( \pi_4 \) &   \( \pi_5 \) & \( \pi_6 \) & \( \pi_7 \) & \( \pi_8 \) & \( \pi_9 \) & \( \pi_{10} \) & \( \pi_{11} \) & \( \pi_{12} \) & \( \pi_{13} \) & \( \pi_{14} \) & \( \pi_{15} \) \\
       \hline
         \( S^0 \) & \(0 \) & 0 & 0 & 0 & 0 & 0 & 0 & 0 & 0 & 0 & 0 & 0 & 0 & 0 & 0 \\
        \( S^1 \) & \( \Z \) & 0 & 0 & 0 & 0 & 0 & 0 & 0 & 0 & 0 & 0 & 0 & 0 & 0 & 0 \\
       \( S^2 \) & 0 & \( \Z \) & \( \Z \)  & \( \Z_2 \)  & \( \Z_2 \)  & \( \Z_{12} \)  & \( \Z_{2} \)  & \( \Z_2 \)  & \( \Z_3 \)  & \( \Z_{15} \)  & \( \Z_2 \)  & \( \Z_2^2 \)  & \( \Z_{12} \times \Z_2\)  & \( \Z_{84}\times\Z_2^2 \)  & \( \Z_2^2 \) \\
        \( S^3 \) & 0 & 0 & \( \Z \) & \( \Z_2 \)  & \( \Z_2 \)  & \( \Z_{12} \)  & \( \Z_2 \)  & \( \Z_2 \)  & \( \Z_3 \)  & \( \Z_{15} \)  & \( \Z_2 \)  & \( \Z_2^2 \)  & \( \Z_{12}\times \Z_2 \)  & \( \Z_{84}\times \Z_2^2 \)  & \( \Z_2^2 \)  \\
        \( S^4 \) &0  & 0 & 0 & \( \Z \) & \( \Z_2 \)  & \( \Z_2 \)  & \( \Z\times\Z_{12} \)  & \( \Z_2^2 \)  & \( \Z_2^2 \)  & \( \Z_{24}\times \Z_3 \)  & \( \Z_{15} \)  & \( \Z_2 \)  & \( \Z_2^3 \)  & \( \Z_{120}\times\Z_{12}\times\Z_2 \)  & \( \Z_{84}\times\Z_2^5 \)  \\
        \( S^5 \) & 0 & 0 & 0 & 0 & \( \Z \) & \( \Z_2 \)  & \( \Z_2 \)  & \( \Z_{24} \)  & \( \Z_2 \)  & \( \Z_2 \)  & \( \Z_2 \)  & \( \Z_{30} \)  & \( \Z_2 \)  & \( \Z_2^3 \)  & \( \Z_{72}\times\Z_2 \)  \\
         \( S^6 \) & 0 & 0 & 0 & 0 & 0 & \( \Z \) & \( \Z_2 \)  & \( \Z_2 \)  & \( \Z_{24} \)  & 0 & \( \Z \)  & \( \Z_2 \)  & \( \Z_{60} \)  & \( \Z_{24}\times\Z_2 \)  & \( \Z_2^3 \) \\
             \( S^7 \) & 0  & 0 & 0 & 0 & 0 & 0 & \( \Z \) & \( \Z_2 \)  & \( \Z_2 \)  & \( \Z_{24} \)  & \( 0 \)  & 0 & \( \Z_2 \)  & \( \Z_{120} \)  & \( \Z_2^3 \) \\
         \( S^8 \) & 0 & 0 & 0 & 0 & 0 & 0 & 0 & \( \Z \) & \( \Z_2 \)  & \( \Z_2 \)  & \( \Z_{24} \)  & 0 & 0 & \( \Z_2 \)  & \( \Z\times\Z_{120} \)  
    \end{tabular}}
    \caption{Low dimensional homotopy groups of spheres. Adapted from https://en.wikipedia.org/wiki/Homotopy\_groups\_of\_spheres.}
    \label{tab:groups}
\end{table}

In \cref{tab:groups} we can see some of the low dimensional homotopy groups
of spheres. According to what we have just showed, we remark that the
only non-torsion groups are exactly where we expect them to be in
the table.
Computing
all of these groups goes well beyond the scope of this paper, but it
is worth mentioning that the approach we have taken here with spectral
sequences can be used to compute some of these groups. For example,
the group \( \pi_4(S^3)=\Z_2 \) can be computed using the fibration
\( f:S^3\rightarrow K(\Z, 3) \), with \( f \) the generator as before.
Spectral sequences also appear once one wishes to compute \emph{stable
homotopy groups of spheres}, most notably through the Adams spectral
sequence and the Adams-Novikov spectral sequence. Indeed, in studying
homotopy groups of spheres, spectral sequences provide an invaluable
tool. We should also mention that there are other approaches to computing the rational homotopy
groups of spheres which don not use spectral sequences,
see for example \cite[]{KlMa04}, and which develop other machinery for
the proof. 

A natural next step in studying homotopy groups of spheres might
be to try to understand the torsion part of the low dimensional groups
in \cref{tab:groups}. As mentioned above, we can use spectral sequences
for some of them, but other groups have to be computed by other
methods, and the interested reader is referred to Chapter 4 of
\cite[]{Hat02} for a treatment of some of these.
\end{document}
