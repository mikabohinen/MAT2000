\documentclass[../main.tex]{subfiles}
\graphicspath{{\subfix{./figures/}}}

\begin{document}
This chapter introduces \emph{spectral sequences}
and uses the cohomological version of the \emph{Serre spectral sequence} to 
compute the cohomology algebra of certain Eilenberg-MacLane spaces, more
specifically, the cohomology algebra of \( K(\Z, n) \) for \( n \in \N \). This
will then provide the final ingredient in our
proof of the rational homotopy groups of spheres.

Our account of spectral sequences will closely follow 
\cite[]{Le00} and \cite[]{Ramos}, which the reader should consult for a more thorough
introduction.
\section{Brief introduction to spectral sequences}
Intuitively a spectral sequence is a collection of `pages', which
are two-dimensional lattices of abelian groups, together with
differential maps between these such that each page is the (co)homology
of the previous one. The abelian group structure of each of these pages
can be captured by the following notion.
\begin{definition}[Bigraded abelian group]
    A \emph{bigraded abelian group} \( E_r=E_r^{*, *} \) is a doubly
    indexed direct sum of abelian groups,
    \begin{equation}
        E_r^{*, *} = \bigoplus_{p,q\in\Z} E_r^{p, q}
    \end{equation}
    where the ordered pair \( (p, q) \) is called the \emph{bidegree} of
    the abelian group \( E_r^{p, q} \).
\end{definition}
We shall only be interested in the first quadrant,
i.e., the bidegrees \( (p, q) \) where \( p,q\geq 0 \), and assume
that the groups \( E_r^{p,q} \) are zero otherwise. We will refer to
such a group as a \emph{first quadrant bigraded abelian group}. There
is also a notion of maps between bigraded abelian groups.
\begin{definition}[]
    A morphism \( d:A\rightarrow B \) between two bigraded abelian groups
    \( A \) and \( B \) in bidegree \( (m,n) \) is a collection of
    group homomorphisms \[ \{d^{p, q}:A^{p, q}\rightarrow B^{p+m, q+n}\}_{p,q\in\Z} \].
\end{definition}
We can now give a more formal description of the cohomological version of a spectral sequence.
\begin{definition}[Spectral sequence]
    A spectral sequence \( E \) is a sequence of bigraded abelian groups
    \( E_r \) and endomorphisms \( d_r:E_r\rightarrow E_r \) in bidegree
    \( (r, r-1) \) for \( r\geq 1 \). Furthermore, each \( d_r \) is
    a differential in the sense that \( d_r\circ d_r =0 \) and, letting
   \begin{equation}
       H^{p,q}(E_r) := \frac{\text{ker}(d^{p,q}_r)}{\text{im}(d^{p-r,q+r-1}_r)},
   \end{equation} 
    we require that \( E_{r+1}^{*,*}\cong H^{*, *}(E_r) \). In other
    words, \( E_{r+1} \) is the cohomology of \( E_r \). 
\end{definition}
For a sequence of first quadrant bigraded abelian groups, there is also
the notion of a \emph{first quadrandt spectral sequence} which is
just an ordinary spectral sequence \( E \) which has trivial components
\( E^{p,q}_r=0 \) for all \( p,q < 0 \) and all \( r\geq 0 \).

If there exists some \( n\geq 1 \) such that \( d^r=0 \) for all \( r\geq n \) we say that the spectral sequence collapses at
the \( n \)--th term. In this case we get that \( E_{r+1}\cong E_r \)
for all \( r\geq n \), and so one has some semblance of convergence of
a spectral sequence. To make this notion more precise, we need to
introduce the concept of a \emph{filtration}.
\begin{definition}[Filtration of an abelian group]
    Given an abelian group \( G \) a filtration \( F^*G \) is a
    sequence of subgroups \( F^iG\subset G \) indexed over the natural numbers
    such that \( F^j \subset F^i \) for all \( j\geq i \) and \( F^0G=G \).
When \( G \) is graded we also require that \( F^i G \) be graded for all
\( i \).
\end{definition}
Analogous to ordinary analysis there is also a concept of a \emph{limiting
term} \( E_\infty \) of a spectral sequence. This comes about by first
noticing that for any spectral sequence \( E \) we have a sequence
of sub-objects \[ 0=B_0\subset B_1 \subset \cdots\subset B_r \subset
\cdots\subset Z_r\subset\cdots\subset Z_1\subset Z_0=E_1 \] where
\( B_r \) and \( Z_r \) are defined inductively such that \( Z_r/B_{r-1}
=\text{ker}(d_r)\)
and \( B_r/B_{r-1}=\text{im}(d_r) \) with \( Z_0:=E_1 \) and \( B_0:=0 \). We then set \( Z_\infty = \cap Z_r \) and \( B_\infty=\cup B_r \).

\begin{definition}[Limiting term]
    Given a spectral sequence \( E \) the limiting term of \( E \),
    denoted \( E_\infty \), is given by
    \begin{equation}
        E_\infty = \frac{Z_\infty}{B_\infty}.
    \end{equation}
\end{definition}
\begin{definition}[Convergence of spectral sequence]
    Given a spectral sequence \( E \) and a graded abelian group
    \( A^* \) with a filtration \( F^*A^* \) we
    say that the spectral sequence \( E \) converges to \( A^* \) if
    for all bidegrees \( (p, q) \) there is an isomorphism 
    \( E_\infty^{p,q}\cong F^pA^{p+q}/F^{p+1}A^{p+q} \). We also
    write this as
    \begin{equation}
        E^{p,q}_r \Longrightarrow A^{p+q}.
    \end{equation}
\end{definition}
If a spectral sequence \( E \) converges to some graded abelian group
\( A^* \) then there is the question of how to deal with the
extension problems to determine what \( A^{p+q} \)
is. However, we are only going to deal with vector spaces over \( \Q \)
and so the short exact sequence
\begin{equation}
    \begin{tikzcd}
        0 & {F_{i-1}A^*} & {F_iA^*} & {F_iA^*/F_{i-1}A^*} & 0
        \arrow[from=1-1, to=1-2]
        \arrow[from=1-2, to=1-3]
        \arrow[from=1-3, to=1-4]
        \arrow[from=1-4, to=1-5]
    \end{tikzcd}
\end{equation}
always splits. That is to say, \( A^{n}\cong\bigoplus_{p+q=n}E^{p,q}_\infty \).

\section{Spectral sequence of algebras}
Since we are not only interested in the cohomology of Eilenberg-MacLane
spaces as vector spaces but also as algebras over \( \Q \), it would
be nice if the spectral sequences could capture this. 

\begin{definition}[Spectral sequence of algebras]
   A \emph{spectral sequence of algebras} is a spectral sequence \( E \) 
   such that each page \( (E_r, d_r) \) is a bigraded differential algebra.
   That is to say, for all \( r\geq 1 \) we have that \( d_r \) is
   an antiderivation, and there is a product
   \( E_r \otimes E_r\rightarrow E_r \) such that the product of
   \( E_{r+1} \) coincides with the one induced by taking the
   cohomology of \( E_r \). 
\end{definition}
With the additional structure of a product operation, we also need to
impose an extra condition on what it means for a spectral sequence of
algebras to converge compared to the ordinary convergence of spectral
sequences. This is because if a spectral sequence of algebras \( E  \)
converges to some filtered algebra \( A^* \) then we inherently have
two products, the one induced from \( E_1,E_2\cdots \) on \( E_\infty \)
as well as the one induced from the filtration of \( A^* \).
\begin{definition}[Filtration of graded differential algebra]
    Given a graded differential algebra \( A^* \) a filtration,
    \( F^*A^* \), is an ordinary filtration of \( A^* \) as a
    graded abelian group such that \( F^{m}A^*\cdot F^{n}A^*\subset
    F^{m+n}A^*\).
\end{definition}
There is also a canonical induced product for
    \( [a]\in F^{p}A^{p+q}/F^{p+1}A^{p+q} \) and \( [b]\in F^{p'}A^{p'
    + q'}/F^{p'+1}A^{p' + q'} \) defined by
\begin{equation}
        \label{eq:filtration}
        [a]\cdot [b] = [a\cdot b] \in F^{p+p'}A^{p + q + p' + q'}/F^{p
        +p' + 1}A^{p + q + p' + q'}
\end{equation}
which is well defined because of the compatibility of the filtration
with the product.

\begin{definition}[Convergence]
    A spectral sequence of algebras, \( E \), is said to converge to
    the graded differential algebra \( A^* \), with a filtration
    \( F^*A^* \), if it converges to \( A^* \) as a spectral sequence
    and the product induced on the \( E_\infty \) page coincides with
    the one in \cref{eq:filtration}.
\end{definition}

\section{The Leray-Serre spectral sequence}
The \emph{cohomological Leray-Serre spectral sequence} is the primary computational tool
needed to compute the cohomology algebra of \( K(\Z, n) \) and was
discovered by Jean-Pierre Serre in his doctoral thesis~\cite[]{Ser51}.
To fully state it, we briefly need to discuss a \emph{system of local
coefficients} on homology, which can be interpreted as a covariant
functor.
\begin{definition}[System of local coefficients]
    Given a space \( X \) and a commutative ring \( R \) a system of local coefficients is a covariant
    functor \[\mathbb{\mathcal{L}}(X;R):\Pi_1(X)\rightarrow \textbf{Mod}_R \]
    from the fundamental groupoid of \( X \) to the category of modules
    over \( R \).
\end{definition}
Given any space \( X \) with a system of local coefficients \( \mathcal{L}(X;R)\),
for some commutative ring \( R \), there is then a modified version of the ordinary singular complex,
which is defined by
\begin{equation}
    C_n(X,\mathcal{L}(X;R)) = \bigoplus_{\sigma\in C_n(X)} \mathcal{L}(
    X;R)(\sigma(1,0,\ldots,0)).
\end{equation}
This also induces a cohomology theory with local coefficients
obtained by dualizing the homology with local coefficient.

We are now able to state the cohomological Leray-Serre
spectral sequence as it appears in \cite[]{Le00} .
\begin{theorem}[Cohomological Leray-Serre spectral sequence]
    \label{thm:sss}
   Let \( R \) be a commutative ring with unit and 
   \( F\rightarrow E \rightarrow B \) a Serre fibration with \( B \) path-connected.
   There is then a first quadrant cohomological spectral sequence of
   algebras \( E \) converging to \( H^*(E;R) \) as an algebra with
   \( E_2^{p,q}\cong H^p(B,\mathcal{H}^q(F;R)) \) where we consider
   \( \mathcal{H}^q(F; R) \) as a system of local coefficients in the
   fibre \( F \). Moreover, this spectral sequence is natural with
   respect to fibre-preserving maps of Serre fibrations. Finally,
   the cup product \( \smile \) on \( H^p(B,\mathcal{H}^q(F;R)) \) is
   related to the product \( \cdot_2:E_2\otimes E_2\rightarrow E_2 \) 
   by \( u\cdot_2 v = (-1)^{p'q}u\smile v \) for \( u\in E^{p,q}_2 \) 
   and \( v\in E_2^{p',q'} \).
\end{theorem}
\begin{proof}
    See \cite[]{Le00}.
\end{proof}
Since \( \pi_1(B) \) induces an action on \( H^*(F) \) in a fibration, the
following proposition will be helpful when the action is trivial.
\begin{proposition}
    \label{prop:coefficients}
   Let \( F\rightarrow E \rightarrow B \) be a Serre fibration and
   \( R \) a commutative ring.
   Suppose \( \pi_1(B) \) acts trivially on \( H^*(F) \), then in
   the Leray-Serre spectral sequence, we have that
   \begin{equation}
       E^{p, q}_2 \cong H^p(B; H^q(F;R)).
   \end{equation}
    Moreover, if \( R=k \) is a field, then
    \begin{equation}
        E^{p, q}_2 \cong H^p(B;k)\otimes_k H^q(F;k).
    \end{equation}
\end{proposition}
\begin{proof}
See \cite[]{Hat04}.
\end{proof}


\section{The rational cohomology algebra of Eilenberg-MacLane spaces}
\begin{definition}[Exterior algebra on a single generator]
   Let \( k \) be a field. The \emph{exterior algebra} on a single
   generator \( x \) over \( k \) is defined as
   \begin{equation}
       \Lambda_k(x)=\frac{k[x]}{(x^2)}.
   \end{equation}
\end{definition}
\begin{theorem}
\label{thm:eil-mac}
The rational cohomology algebra of \( K(\Z, n) \)
is isomorphic to
an exterior algebra over \( \Q \)  on a single generator in degree \( n \)
if \( n \) is odd, and a free graded commutative algebra over \( \Q \) 
on a single generator if \( n \) is even.
    
\end{theorem}
\begin{proof}
We proceed by induction. For \( n=1 \) it is undoubtedly true that
the rational cohomology of \( K(\Z,1)\simeq S^1 \) is an exterior algebra on a generator in
degree one. Therefore, we assume that the induction hypothesis holds
for \( n-1 \) when \( n\geq 1 \) and want to show that this implies
the statement for \( n \). Assume first that \( n \) is even so that
\( n-1 \) is odd. We now take the path-space fibration of \( K(\Z, n) \)
and get
\begin{equation}
    \begin{tikzcd}
        {K(\Z,n-1)} & {*} & {K(\Z, n)}
        \arrow[from=1-1, to=1-2]
        \arrow[from=1-2, to=1-3]
    \end{tikzcd}
\end{equation}
Now, since \( \pi_1(K(\Z, n))=0 \), we get from \cref{prop:coefficients} that the \( E_2 \)--page of
the Serre spectral sequence takes the form
\begin{equation}
    E^{p,q}_2 
    \cong H^p(K(\Z, n); \Q) \otimes H^q(K(\Z, n-1); \Q).
\end{equation}
By assumption we have that \( H^*(K(\Z, n-1); \Q)\cong \Lambda_\Q(y) \),
where \( \abs{y}=n-1 \), and so we have
\begin{equation}
    H^q(K(\Z, n-1); \Q) =
    \begin{cases}
        \Q, &\text{ if }q=0, n-1\\
        0, &\text{ else.}
    \end{cases}
\end{equation}
Since \( * \) has trivial cohomology, the only component that survives
is \( E_2^{0, 0} \) while all the other on the \( E_2 \)--page must
die. This implies that \( E_n^{0, n-1}\cong E_2^{0, n-1}\cong \Q_y \) cannot survive to
the next page and
so the differential \( d_n^{0, n-1}:E_2^{0, n-1}\rightarrow E_2^{n, 0} \)
must be injective. This ensures that \( E^{n, 0}_n\cong E^{n,0}_2 \) contains at
least a \( \Q \). Furthermore, since \( d_n^{n,0}=0 \) it also cannot
contain more than \( \Q \), and so \( E^{n,0}_n\cong E^{n,0}_2\cong\Q \).
We then have that \( d_n^{0, n-1} \) is an isomorphism and so we set
\( x=d_n^{0, n-1}(y) \) to be the generator of \( E^{n,0}_2 \). From
this we immediately have that 
\begin{align*}
    E^{n, n-1}_n &\cong E^{n,n-1}_2 \\
                 &\cong \Q_x\otimes \Q_y \\
                 &= \Q_{xy}.
\end{align*}
Moreover, by similar reasoning as before, we must have that \( d^{n, n-1}
_n\) is an isomorphism. Indeed, we have that \( E^{0, kn}_2 \cong\Q \) 
for all \( k\geq 0 \). Furthermore, by the graded differential
structure, we see that inductively
\begin{align*}
    d_n^{kn, n-1}(x^k y)&= d_n^{kn, 0}(x^k)y + (-1)^{\abs{x^k}}x^k d_n^{0, n-1}(y) \\
              &= 0 + x^k\cdot x \\
              &= x^{k+1}.
\end{align*}
Thus the generator of \( E_2^{kn, 0} \) is \( x^k \) and so we finally
get that
\begin{equation}
    H^*(K(\Z, n);\Q)\cong E^{*, 0}_2 \cong \Q[x]
\end{equation}
which concludes the proof for even \( n \).

Assume now that \( n \) is odd. We then have that \( H^*(K(\Z,n-1))\cong
\Q[y]\) with \( \abs{y}=n-1 \). Applying the same path-space fibration as before we have that
\( E_2^{p, q}\cong H^p(K(\Z, n);\Q)\otimes H^q(K(\Z, n-1); \Q)\). By
similar reasoning as in the even case we must have that \( d_n^{0, n-1} \)
is an isomorphism. We therefore set \( x=d^{0, n-1}_n(y) \) to be the
generator of \( E_n^{n, 0}\cong\Q \). Consequently, we have that
\( xy^k \) is a generator of \( E_2^{n, k(n-1)}\cong E_n^{n, k(n-1)}\cong
\Q_{xy^k}\) for all \( k\geq 0 \). The claim now is that
\( E_n^{q, 0}=0 \) for all \( q>n \). To see this, note first that
we inductively have
\begin{align*}
    d_n^{0, k(n-1)}(y^k)&=d_n^{0, n-1}(y)y^{k-1}+(-1)^{\abs{y}}yd_n^{0, (k-1)(n-1)}(y^{k-1})\\
    &= xy^{k-1} +y(k-1)xy^{k-2} \\
    &= kxy^{k-1}
\end{align*}
which means that \( xy^{k-1} \) is killed by \( \frac{1}{k}y^k \).
However, this means that \( d^{0, k(n-1)}_n \) is an isomorphism
for all \( k \geq 1 \), and so we must have that \( E_n^{2n, *}=0 \) for
degree reasons. We then inductively have that \( E_n^{q, *}=0 \) for
\( q>n \), which
implies that
\begin{align*}
    H^*(K(\Z, n);\Q)&\cong E_2^{*, 0}\\
                    &=E_n^{*, 0}\\
                    &\cong\Q[x]/(x^2) \\
                    &= \Lambda_\Q(x).
\end{align*}
Hence we see that the rational cohomology algebra of \( K(\Z, n) \), for
odd \( n \), is an exterior algebra on one generator in degree \( n \)
and so we are done.
\end{proof}
\end{document}
