\documentclass[../main.tex]{subfiles}
\graphicspath{{\subfix{./figures/}}}
\begin{document}
In this chapter our main goal will to be develop all the tools we need
for what is to come. We start out by defining homotopy groups inductively
as loop spaces before moving on to some other stuff.

\section{Homotopy groups as loop spaces}
One normally defines the homotopy group \( \pi_n(X,p) \) of a pointed
topological space \( (X,p) \) to be the homotopy classes of maps
\( f:(I^n, \partial I^n)\rightarrow (X, p) \). We shall take a slightly
different approach in our definition, but one could just as well have used
this definition and none of the results would change.

To start off we need the concept of a \emph{loop space} as it appears in
\cite{BoTu82} .
\begin{definition}[Loop space]
   For a topological space \( X \) and a point \( p \in X \)
   we define the loop space of \( X \)
   to be the space \( \Omega_p X \) which consists of continuous maps \( f \)
   from the unit circle \( S^1 \)(realized as the set \( \{z \in 
   \C : \abs{z}=1\}  \)) into X such that \( f(1)=p \). In order for
   \( \Omega_p X \) to be a topological space we equip it with the
   compact open topology.
\end{definition}
With the definition of a loop space we can then use this to inductively
define \( \pi_n(X, p) \) for arbitrary \( n \).
\begin{definition}[Homotopy groups]
   The \( 0 \)th homotopy group \( \pi_0 (X, p) \) of a pointed topological space
   is defined to be its path components. For the inductive step
   we set the \( n+1 \)th homotopy group to be \( \pi_{n+1}(X,p) = \pi_n(\Omega_{p}X, \overline{p})\)
   where \( \overline{p} \) is the constant map to the point \( p \).
\end{definition}
With the operation of composition of loops this turns the homotopy
groups into actual groups where the constant map acts as the identity,
and the inverse is just the reversal of each loop.


\section{Suspension and the smash product}
With a firm grasp on the definition of homotopy groups we can begin
to introduce some of the main tools and concept we shall need in
studying them. Two of the most basic of these is the \emph{suspension} of a
space as well as the \emph{smash product} of two spaces. We shall define both
of these as they appear in chapter 0 of \cite{Hat02}.
\begin{definition}[Suspension]
    Given a space \( X \) the suspension of it, denoted \( SX \), is
    the quotient of \( X\times I \) where one collapse both \( X\times
    \{0\}\) and \( X\times \{1\} \) into one point each.
\end{definition}
The prototypical example of a suspension is pictured in Figure~\ref{fig:
suspension} where we have taken the suspension of the unit circle \( S^1 \)
and see how it is homeomorphic to \( S^2 \). One can also take the
suspension of a map \( f:X\rightarrow Y \) to be \( Sf:SX\rightarrow
SY\) defined as the quotient of \( f\times \text{id}:X\times I \rightarrow
Y \times I\). It is also quite straightforward to check that
\( S \) respects composition
and we thus have that \( S \) is actually a covariant functor.

With the suspension out of the way we can now define the smash product
of two spaces.
\begin{definition}[Smash product]
    Given two pointed spaces \( (X, x_0) \) and \( (Y, y_0) \) we define
    the smash product of
    them, written \( X \wedge Y \), to be the quotient \( X \times
    Y/X\vee Y\) where the wedge \( X \vee Y \subset X\times Y\) is
    understood to be the part that contains the point \( (x_0, y_0) \).
\end{definition}
The smash product may be a bit harder to visualize at first compared
to the suspension. However, trying out on spheres provides useful
examples as it can be shown in general that \( S^n \wedge S^m
\cong S^{n+m}\).

Both of the operators defined above are operators which
plays an integral role in the theory of higher homotopy groups. We shall
frequently encounter these in the following chapters.

\section{Serre fibrations and fiber bundles}
Another crucial component in the study of homotopy groups of spheres is
long exact sequences of said groups. To this end we need to be able
to understand when these can occur. As we shall see, this is the case
when we have a \emph{Serre fibration}. In order to understand the
definition of a Serre fibration we first need some preliminary notions,
more specifically, the homotopy lifting property. We shall define this
as it appears in \cite{Mit01}.

\begin{definition}[Homotopy lifting property]
   Given spaces \( X \), \( E \), and \( B \), together with a map
   \( p:E\rightarrow B \), we say that \( p \) has the homotopy lifting
   property if
      \begin{enumerate}[label=(\roman*)]
         \item for any homotopy \( F:X\times I \rightarrow B \) and
         \item an initial map \( \tilde{F}_0:X\rightarrow E \) such
            that \(F_0 = p\circ \tilde{F}_0 \)
      \end{enumerate}
   there exist a map \( \tilde{F} \) making the diagram in \cref{fig:homotopy-lifting-property}
   commute.
\end{definition}
\begin{figure}
   \centering
   \begin{tikzcd}
       X & {} & E \\
       \\
       {X\times I} && B
       \arrow["{\tilde{F}_0}", from=1-1, to=1-3]
       \arrow["F"', from=3-1, to=3-3]
       \arrow["p", from=1-3, to=3-3]
       \arrow["{i_0}"', from=1-1, to=3-1]
       \arrow["{\tilde{F}}", from=3-1, to=1-3]
   \end{tikzcd}
   \caption{The map \( i_0 \) is taken to be the identification of
   \( X \) with \( X\times \{0\}  \) in \( X\times I \).}
   \label{fig:homotopy-lifting-property}
\end{figure}

With this definition in mind we say that \( p:E\rightarrow B \) is a
Serre fibration if \( p \) has the homotopy lifting property for all
CW-complexes. One also has the notion of a general \emph{fibration} in
which the requirement is that \( p \) must have the homotopy extension
property for all spaces and not just the CW-complexes.

An important class of Serre fibrations are \emph{fiber bundles} which
we now shall now define.
We shall define a fiber bundle as it appears in \cite[p.~376--377]{Hat02}. To start off we need to understand the notion of a fiber.
\begin{definition}[Fiber]
    \label{def:fiber} 
    For spaces \( E \) and \( B \), a surjective map \( p:E\rightarrow B \),
    and a point \( b \in B \) we shall call the set \( p^{-1}(b) \)
    the \emph{fiber} of \( b \). If there exists a space \( F \) such
    that for all points \( b\in B \) we have \( p^{-1}(b)\cong F \) then
    we simply refer to \( F \) as the fiber of \( B \).
\end{definition}
We now just need one more technical notion before introducing the
definition of a fiber bundle.
\begin{definition}[Local trivialization]
    Let \( F \), \( E \), \( B \) be spaces with a map \( p:E\rightarrow B \)
    such that \( F \) is a fiber of \( B \). If there exists an open
    covering \( \{U_i\} \) of \( B \) such that for all \( U_i \) there
    is a corresponding homeomorphism \\ \( \varphi_i:p^{-1}(U_i)\rightarrow
    U_i\times F\) which makes the diagram in \cref{fig:local-triv}
    commute, then the collection \( \{(U_i, \varphi_i)\} \) is called
    a \emph{local trivialization} of the spaces.
\end{definition}
With that out of the way we are now ready for the definition of a fiber
bundle.
\begin{definition}[Fiber Bundle]
    \label{def:fiber-bundle}
    A fiber bundle, denoted \( (E, B, p, F) \), is a
    structure consisting of spaces \( E \), \( B \), and a fiber
    \( F \) of \( B \), with the corresponding surjection \( p :E\rightarrow B \).
    The space \( E \) is called the \emph{total space}, \( B \) is
    called the \emph{base space}, and the map 
    \( p \) is called the \emph{projection map}. We furthermore
    require that there exists a local trivialization of the spaces
    in the bundle.
    Another way to denote a fiber bundle is as
    \begin{equation}
        \label{eq:exact-seq}
        \begin{tikzcd}
            F \arrow[r]  & E \arrow[r, "p"]  & B
        \end{tikzcd}
    \end{equation}
\end{definition}
\begin{figure}[b]
    \centering
    \begin{tikzcd}
    p^{-1}(U) \arrow[rdd, "p"'] \arrow[rr, "\varphi_i"] &   & U\times F \arrow[ldd] \\
                                            &   &                       \\
                                            & U &   
    \end{tikzcd}
    \caption{The unlabeled map is the projection onto the first factor.}
    \label{fig:local-triv}
\end{figure}

Our claim now is that fiber bundles are examples of Serre fibrations.

\begin{proposition}
   A fiber bundle \( (F, E, p, B) \) satisfy the homotopy lifting
   property for all spaces \( X \).
\end{proposition}
\end{document}
