\documentclass[../main.tex]{subfiles}
\graphicspath{{\subfix{./figures/}}}
\begin{document}

\section{Homotopy groups as loop spaces}
One often defines the homotopy group \( \pi_n(X,p) \) of a pointed
topological space \( (X,p) \) to be the homotopy classes of maps
\( f:(I^n, \partial I^n)\rightarrow (X, p) \). However, we shall adopt
the approach taken in \cite[]{BoTu82} and define them in terms of 
\emph{loop spaces}.

\begin{definition}[Path space]
Given a based space \( (X, p) \), the \emph{path space} \( PX=X^I \) is the set
of based maps from the unit interval \( I=[0,1] \) to \( X \) endowed
with the compact open topology such that \( f(0)=p \).
\end{definition}
There is a canonical map \( \chi:PX\rightarrow X \) given by \( \chi(f)=f(1) \). The loop space of \( X \) can then be described as a fibre
of this map.

\begin{definition}[Loop space]
   For a based space \( (X, p) \),
   we define the loop space of \( X \)
   as the space \( \Omega X=\chi^{-1}(p) \) endowed with the subspace
   topology.
\end{definition}
\begin{definition}[Homotopy groups]
   \label{def:homotopy-groups}
   The \( 0 \)th homotopy group, \( \pi_0 (X, p) \), of a pointed topological space
   is defined to be its path components. For the inductive step,
   we set the \( n+1 \)th homotopy group to be \( \pi_{n+1}(X,p) = \pi_n(\Omega_{p}X, \overline{p})\),
   where \( \overline{p} \) is the constant map to the point \( p \).
\end{definition}
The operation of the composition of loops turns the homotopy
groups into actual groups where the constant map acts as the identity,
and the inverse is just the reversal of each loop.



\iffalse \section{Suspension and the smash product}
With a firm grasp on the definition of homotopy groups we can begin
to introduce some of the main tools and concept we shall need in
studying them. Two of the most basic of these is the \emph{suspension} of a
space as well as the \emph{smash product} of two spaces. We shall define both
of these as they appear in chapter 0 of \cite{Hat02}.
\begin{definition}[Suspension]
    Given a space \( X \) the suspension of it, denoted \( SX \), is
    the quotient of \( X\times I \) where one collapse both \( X\times
    \{0\}\) and \( X\times \{1\} \) into one point each.
\end{definition}
The prototypical example of a suspension is pictured in Figure~\ref{fig:
suspension} where we have taken the suspension of the unit circle \( S^1 \)
and see how it is homeomorphic to \( S^2 \). One can also take the
suspension of a map \( f:X\rightarrow Y \) to be \( Sf:SX\rightarrow
SY\) defined as the quotient of \( f\times \text{id}:X\times I \rightarrow
Y \times I\). It is also quite straightforward to check that
\( S \) respects composition
and we thus have that \( S \) is actually a covariant functor.

With the suspension out of the way we can now define the smash product
of two spaces.
\begin{definition}[Smash product]
    Given two pointed spaces \( (X, x_0) \) and \( (Y, y_0) \) we define
    the smash product of
    them, written \( X \wedge Y \), to be the quotient \( X \times
    Y/X\vee Y\) where the wedge \( X \vee Y \subset X\times Y\) is
    understood to be the part that contains the point \( (x_0, y_0) \).
\end{definition}
The smash product may be a bit harder to visualize at first compared
to the suspension. However, trying out on spheres provides useful
examples as it can be shown in general that \( S^n \wedge S^m
\cong S^{n+m}\).

Both of the operators defined above are operators which
plays an integral role in the theory of higher homotopy groups. We shall
frequently encounter these in the following chapters.
\fi

\section{Serre fibrations}
Another crucial component of studying homotopy groups of spheres are
long exact sequences of said groups. To this end, we need
to understand when these can occur. As we shall see, this is the case
when we have a \emph{Serre fibration}. In order to understand the
definition of a Serre fibration, we first need some preliminary notions,
more specifically, the homotopy lifting property. We shall define this
as it appears in \cite{Mit01}.

\begin{definition}[Homotopy lifting property]
   Given spaces \( X \), \( E \), and \( B \), together with a map
   \( p:E\rightarrow B \), we say that \( p \) has the homotopy lifting
   property if
      \begin{enumerate}[label=(\roman*)]
         \item for any homotopy \( F:X\times I \rightarrow B \) and
         \item an initial map \( \tilde{F}_0:X\rightarrow E \) such
            that \(F_0 = p\circ \tilde{F}_0 \)
      \end{enumerate}
   there exists a map \( \tilde{F} \) making the diagram in \cref{fig:homotopy-lifting-property}
   commute.
\end{definition}
\begin{figure}
   \centering
   \begin{tikzcd}
       X & {} & E \\
       \\
       {X\times I} && B
       \arrow["{\tilde{F}_0}", from=1-1, to=1-3]
       \arrow["F"', from=3-1, to=3-3]
       \arrow["p", from=1-3, to=3-3]
       \arrow["{i_0}"', from=1-1, to=3-1]
       \arrow["{\tilde{F}}", from=3-1, to=1-3]
   \end{tikzcd}
   \caption{The map \( i_0 \) is taken to be the identification of
   \( X \) with \( X\times \{0\}  \) in \( X\times I \).}
   \label{fig:homotopy-lifting-property}
\end{figure}

We say that \( p:E\rightarrow B \) is a
Serre fibration if \( p \) has the homotopy lifting property for all
CW-complexes. One also has the notion of a general \emph{fibration} in
which the requirement is that \( p \) must have the homotopy extension
property for all spaces and not just the CW-complexes.

For our purposes, a fundamental property of fibrations is that they admit
a long exact sequence of homotopy groups.
\begin{proposition}
  Given a Serre fibration \( p:E \rightarrow B \) with basepoint
  \( b_0\in B \) let \(  F=p^{-1}(b_0) \) and
  \( x_0\in F \). There is then a long exact sequence of homotopy
  groups
  \begin{equation}
     \begin{tikzcd}
          \cdots & {\pi_n(F,x_0)} & {\pi_n(E,x_0)} & {\pi_n(B,b_0)} & {\pi_{n-1}(F,x_0)} & \cdots 
          \arrow[from=1-1, to=1-2]
          \arrow[from=1-5, to=1-6]
          \arrow[from=1-2, to=1-3]
          \arrow["{p_*}", from=1-3, to=1-4]
          \arrow[from=1-4, to=1-5]
      \end{tikzcd}
  \end{equation}

\end{proposition}
\begin{proof}
   See section 4.2 of \cite[]{Hat02}.
\end{proof}

The \emph{path space fibration} of \( X \) gives a canonical class of
fibrations for any space \( X \).

\begin{proposition}[Path space fibration]
   Given any based space \( (X,x_0) \), the map \( \chi:PX\rightarrow X \) is a
   Serre fibration with fibre \( \Omega X \).
\end{proposition}
\begin{proof}
   See \cite[]{Felix}.
\end{proof}

\begin{example}[Hopf fibration]
\label{ex:hopf}
An important historical example of a Serre fibration is the Hopf fibration
\begin{equation}
\label{eq:hopf}
\begin{tikzcd}
	{S^1} & {S^3} & {S^2}
	\arrow[from=1-1, to=1-2]
	\arrow["p", from=1-2, to=1-3]
\end{tikzcd}
\end{equation}
where \( p:S^3 \rightarrow S^2 \) is defined as \( p(z, w)=[z:w] \), 
thinking of \( S^3 \) as the unit sphere in \( \C^2 \) and \( S^2 \) as
\( \C P^1 \). For proof that this is a Serre fibration, the reader is
referred to \cite[]{Hat02}. We know that \( \pi_i(S^n)=0 \) for \( i<n \)
, see \cite[]{Hat02}, and so the only interesting part of the long
exact sequence associated with the Hopf fibration is
\begin{equation}
   \begin{tikzcd}
       0 & {\pi_3(S^3)} & {\pi_3(S^2)} & 0
       \arrow[from=1-1, to=1-2]
       \arrow["{p_*}", from=1-2, to=1-3]
       \arrow[from=1-3, to=1-4]
   \end{tikzcd}
\end{equation}
Thus, \( p_*:\pi_3(S^3)\rightarrow \pi_3(S^2) \) is an
isomorphism. Moreover, by the Hurewicz theorem; \( \pi_3(S^3)\cong\Z \) and
one of the generators is represented by the identity map \( \iota:
S^3\rightarrow S^3\). From this, we get that \( \pi_3(S^2)\cong\Z \) and
that one of the generators are
\begin{align*}
   p_*([\iota]) &= [p\circ \iota] \\
                &= [p],
\end{align*}
which means that the Hopf fibration \( p:S^3\rightarrow S^2 \) represents
the generator of \( \pi_3(S^2) \). The historical importance of this
result is that this showed that homotopy groups behave different compared
to homology groups. So much more care is needed when thinking about
\( \pi_{k+n}(S^n) \) in general. For a more comprehensive discussion
of the Hopf fibration, the reader is referred to \cite[]{Lyo03}.
\end{example}
\section{Homotopy fibres}
One of the main ingredients in our proof for the rational homotopy groups of even
spheres is the use of \emph{homotopy fibres}.
\begin{definition}[Mapping fibration]
Let \( f:Y\rightarrow X \) be a map between spaces. We then define
the set \( P(f) \) to be the fibreed product
\begin{equation}
   P(f) = X^I\times_X Y = \{(\alpha, y)\hspace{1mm} | \hspace{1mm} \alpha(1)=f(y)\}.
\end{equation}
The map \( p:P(f)\rightarrow X \), defined by \( p(\alpha, y)=\alpha(0) \),
then gives the \emph{mapping fibration} of \( f \).
\end{definition}
This gives us a way of associating a fibration to any map.
\begin{proposition}
  Given a map \( f:Y\rightarrow X \), the mapping fibration \(p:P(f)\rightarrow X  \) 
  is a Serre fibration.
\end{proposition}
\begin{proof}
   See \cite[]{DaKi12}.
\end{proof}

\begin{definition}[Homotopy fibre]
   Given a map \( f:Y\rightarrow X \) and a point \( x\in X \), we
   define the homotopy fibre of \( f \) over \( x \) as
   the fibre \( p^{-1}(x) \) of the mapping fibration.
\end{definition}

The fact that the mapping fibration is a Serre fibration lets us view
any map \( f: Y\rightarrow X \) 
as a fibration up to homotopy in light of the following proposition.

\begin{proposition}
   Let \( f: Y\rightarrow X \) be any map, then there is a
   homotopy equivalence \( \phi:Y\rightarrow P(f) \), which makes the
   following diagram commute
   \begin{equation}
      \begin{tikzcd}
          Y & {} & {P(f)} \\
          && {} \\
          && X
          \arrow["f"', from=1-1, to=3-3]
          \arrow["\phi", from=1-1, to=1-3]
          \arrow["p", from=1-3, to=3-3]
      \end{tikzcd}
   \end{equation}
\end{proposition}
\begin{proof}
   Define \( \phi:Y\rightarrow P(f) \) by \( \phi(y)=(c_{f(y)}, y) \)
   where \( c_{f(y)} \) is the constant path at \( f(y) \). Let now
   \( \psi:P(f)\rightarrow Y \) be given by \( \psi(\alpha, y)=y \).
   Then \( \psi\circ\phi = \text{id}_Y \), and so we only need to show
   that \( \phi\circ\psi \simeq \text{id}_{P(f)} \). Now, if
   \( (\alpha, y)\in P(f) \), then \( (\phi\circ\psi)(\alpha, y)=(c_{f(y)}, y) \),
   and so all we need do is find a homotopy between the constant path
   \( c_{f(y)} \) and \( \alpha \). This is easy enough; let
   \( \alpha_t \) be the restriction of \( \alpha \) to the interval
   \( [t, 1] \) with the necessary re-parameterization. We then
   define the homotopy \( F:P(f)\times [0,1]\rightarrow P(f) \) by
   \( F((\alpha, y), t)=(\alpha_t, y) \). Since, by construction, we have
   \( \alpha_1 = c_{f(y)} \) we see that \( F((\alpha, y), 0)=(\alpha,y)=\text{id}_{P(f)}(\alpha, y) \) and \( F((\alpha, y), 1)=(c_{f(y)},y)=(\phi\circ\psi)(\alpha, y) \).
   We thus see that \( \phi \) is a homotopy equivalence, and it is also
   evident that \( \phi \) makes the diagram commute.
   
\end{proof}
Using this fact, we can replace \( P(f) \) with \( Y \) in the long
exact sequence associated with the mapping fibration \( p:P(f)\rightarrow X \).
This, in turn, is going to allow us to compute the homotopy groups
of the homotopy fibre of a homotopy fibre
which in the case of some \emph{Eilenberg-MacLane spaces} shows that
this is a new Eilenberg-MacLane space.
\section{Eilenberg-MacLane spaces}
Eilenberg-MacLane spaces first appeared in \cite[]{EiMa}, and their rational
cohomology will play a crucial role in the  
proof of the rational homotopy groups of spheres. So we shall now
give an account of the critical features of these spaces, which we need
to study the rational homotopy groups of spheres.
\begin{definition}[Eilenberg-MacLane spaces]
   For \( n\in \N \) and \( G \) a group, we say that a space \( X \) 
   is an Eilenberg-MacLane space of type \( K(G, n) \) if its \( n \)th homotopy group is equal
   to \( G \) and otherwise trivial.
\end{definition}
From this definition and \cref{def:homotopy-groups}, it is clear that
\[ \Omega K(G,n)\simeq \Omega K(G, n-1) \] for \( n>1 \). Noting the fact that
\( PK(G, n)\simeq * \) one gets a useful fibration by applying the
path space fibration, which in this case becomes
\begin{equation}
   \begin{tikzcd}
	{K(G,n-1)} & {*} & {K(G,n)}
	\arrow[from=1-1, to=1-2]
	\arrow[from=1-2, to=1-3]
   \end{tikzcd}
\end{equation}
Applying the \emph{Serre spectral sequence} to this fibration
shall allow us to compute the rational cohomology of \( K(\Z, n) \).

Another crucial aspect of Eilenberg-MacLane spaces for our purposes
is its connection to cohomology.
\begin{proposition}
\label{prop:mapping-space}
  For an abelian group \( G \) and \( n \in \N \) there is a natural
  bijection for all CW-complexes \( X \) between the space \( [X, K(G, n)] \), the space of
  homotopy classes of maps from \(X  \) to \( K(G, n) \), and 
  \( H^n(X;G) \), the \( n \)th cohomology group of \( X \) with
  coefficients in \( G \). Endowing \( [X, K(G, n)] \) with a
  canonical group structure this turns into a natural isomorphism.
 
\end{proposition}
\begin{proof}
  See Section 4.2 in \cite[]{Hat02}.
\end{proof}
To conclude this section, let \( f:K(\Z, n)\rightarrow K(\Z, 2n) \) be
any map. If we choose some basepoint \( x_0\in K(\Z, 2n) \), we let
\( A \) denote the homotopy fibre of \( f \) over \( x_0 \) with
induced map \( \alpha:A\rightarrow K(\Z, n) \). We then choose a
basepoint \( a_0\in A \) and let \( B \) be the homotopy fibre of
\( \alpha \) over \( a_0 \) with induced map \( \beta:B\rightarrow A \) .
\begin{proposition}
\label{prop:fibres}
   The homotopy groups \( \pi_k(A) \) are infinite cyclic for
   \( k=n, 2n-1 \), and otherwise trivial. On the other hand,
   \( B \) is an Eilenberg-MacLane space of type
   \( K(\Z, 2n-1) \).
\end{proposition}
\begin{proof}
   We first compute \( \pi_k(A) \) using the long exact sequence
   of the fibration
   \begin{equation}
      \begin{tikzcd}
          A & {} & {K(\Z, n)} && {K(\Z, 2n)}
          \arrow[""{name=0, anchor=center, inner sep=0}, "\alpha", from=1-1, to=1-3]
          \arrow["f", from=1-3, to=1-5]
      \end{tikzcd}
   \end{equation}
   Looking at a section of the exact sequence, we have
   \begin{equation}
      \begin{tikzcd}
          \cdots & {\pi_k(A)} & {\pi_k(K(\Z, n))} & {\pi_k(K(\Z,2n))} & \cdots
          \arrow[from=1-4, to=1-5]
          \arrow["{f_*}", from=1-3, to=1-4]
          \arrow["{\alpha_*}", from=1-2, to=1-3]
          \arrow[from=1-1, to=1-2]
      \end{tikzcd}
   \end{equation}
   From this we see that when \( k=n \) we have \( \pi_{n+1}(K(\Z, 2n))=
   \pi_n(K(\Z, 2n))=0\) and so \( \alpha_* \) is an isomorphism for \( k=n \) 
   and hence \( \pi_n(A)=\Z \). A similar computation shows that
   \( \pi_{2n-1}(A)=\Z \). Now, if \( k\neq n, 2n-1 \) then \( \pi_{k+1}(K(\Z, 2n))=\pi_k(K(\Z, n))=0 \)
   and so, necessarily, \( \pi_k(A)=0 \) showing the result for \( A \).

   Using the homotopy groups of \( A \) we then take a look at the
   fibration given by
   \begin{equation}
      \begin{tikzcd}
          B && A && {K(\Z, n)}
          \arrow["\beta", from=1-1, to=1-3]
          \arrow["\alpha", from=1-3, to=1-5]
      \end{tikzcd}
   \end{equation}
   Again, taking a section of the long exact sequence, we get
   \begin{equation}
      \begin{tikzcd}
          \cdots & {\pi_k(B)} & {\pi_k(A)} & {\pi_k(K(\Z,n))} & \cdots
          \arrow[from=1-1, to=1-2]
          \arrow["{\beta_*}", from=1-2, to=1-3]
          \arrow["{\alpha_*}", from=1-3, to=1-4]
          \arrow[from=1-4, to=1-5]
      \end{tikzcd}
   \end{equation}
   For \(k =2n-1 \) we have that \( \beta_* \) is an isomorphism
   so that \( \pi_{2n-1}(B)\cong\Z \). The only non-obvious part
   of the sequence is
   \begin{equation}
      \begin{tikzcd}
          0 & {\pi_n(B)} & \Z & \Z & {\pi_{n-1}(B)} & 0
          \arrow[from=1-1, to=1-2]
          \arrow["{\beta_*}", from=1-2, to=1-3]
          \arrow["{\alpha_*}", from=1-3, to=1-4]
          \arrow["{\partial_*}", from=1-4, to=1-5]
          \arrow[from=1-5, to=1-6]
      \end{tikzcd}
   \end{equation}
   By exactness we have that 
   \begin{equation}
      \label{eq:exactness}
      \pi_{n-1}(B)\cong \frac{\Z}{\text{im}(\alpha_*)}.
   \end{equation}
   Furthermore,
   \begin{equation}
      \text{im}(\alpha_*)\cong \frac{\Z}{\text{im}(\beta_*)}
   \end{equation}
   and so the only way for the quotient in \ref{eq:exactness} to make
   sense is if \( \text{im}(\beta_*)=0 \), but then \( \pi_n(B)=0 \)
   and \( \pi_{n-1}(B)\cong \Z/\Z \cong 0 \), showing that the only
   non trivial homotopy group is \( \pi_{2n-1}(B)=\Z \). We, therefore,
   must have that \( B \) is an Eilenberg-MacLane space of type
   \( K(\Z, 2n-1) \) which was what we wanted to show and so we are done.
\end{proof}


\section{Rational homotopy equivalence}
As mentioned in the introduction, one of the main ideas for the proof
of \cref{eq:even-spheres} and \cref{eq:odd-spheres} is to construct
a rational homotopy equivalence between the spheres and some other space.
We want to use the algebra structure in cohomology,
so the following results will be essential to us.
\begin{definition}[Rational homotopy equivalence]
    Let \( f:X\rightarrow Y \) be a map. We say that \( f \) is a
    \emph{rational homotopy equivalence} if the induced map on
    rationalized homotopy groups, \( \pi_*(f)\otimes\Q:\pi_*(X)\otimes
    \Q\rightarrow\pi_*(Y)\otimes\Q\), is an isomorphism.
\end{definition}
This weakens the criterion of weak homotopy equivalence, which requires
that \( \pi_*(f):\pi_*(X)\rightarrow\pi_*(Y) \) be an isomorphism. Thus,
every homotopic and every weakly homotopic space is rationally homotopic.
It turns out, due to a result of Serre, that the above definition is
equivalent to requiring that the induced map in homology with rational
coefficients is an isomorphism.
\begin{proposition}
    Let \( f:X \rightarrow Y \) be a map between spaces. Then, \( f \)
    is a rational homotopy equivalence if and only if the induced
    map \( f_*:H_*(X;\Q)\rightarrow H_*(Y;\Q) \) is an isomorphism.
\end{proposition}
\begin{proof}
    See \cite[]{Ser53}.
\end{proof}
We also have a similar result for cohomology, which requires the notion
of a \emph{uniquely divisible group}.
\begin{definition}[Uniquely divisible group]
    An abelian group \( G \) is said to be \emph{divisible} if for all
    \( g \in G \), and all positive integers \( n \), there exists \( y
    \in G\) such that
    \( ny=g \). The group is uniquely divisible if \( y \) is unique.
\end{definition}
\begin{proposition}
   \label{prop:rat-cohomology}
   For any map \( f:X \rightarrow Y \), we have that it is a rational
   homotopy equivalence if and only if for any uniquely divisible
   abelian group \( G \) the induced map \( f^*:H^*(Y;G)\rightarrow 
   H^*(X;G)\) is an isomorphism.
\end{proposition}
\begin{proof}
    See \cite[]{Ber12}.
\end{proof}

\end{document}
